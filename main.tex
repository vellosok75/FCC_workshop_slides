\documentclass[aspectratio=169]{beamer}

\mode<presentation>
{
  \usetheme{default}
  \usecolortheme{seahorse}
  \usefonttheme{professionalfonts}
  \setbeamertemplate{navigation symbols}{}
  \setbeamertemplate{caption}[numbered]
  \setbeamertemplate{footline}[frame number]
}

\usepackage[utf8x]{inputenc}
\usepackage{graphicx}
\usepackage{physics}
\usepackage{caption}
\usepackage{subcaption}
\usepackage{xcolor}
\usepackage{physics}
\usepackage{amsmath}
\usepackage{tikz}
\usepackage{mathdots}
\usepackage{yhmath}
\usepackage{cancel}
\usepackage{color}
\usepackage{siunitx}
\usepackage{array}
\usepackage{multirow}
\usepackage{amssymb}
\usepackage{gensymb}
\usepackage{tabularx}
\usepackage{extarrows}
\usepackage{booktabs}
\usepackage{centernot}
\usetikzlibrary{fadings}
\usetikzlibrary{patterns}
\usetikzlibrary{shadows.blur}
\usetikzlibrary{shapes}



\title{Online tuning of storage ring non-linear dynamics at SIRIUS and fast ORM measurement}

\author{Matheus Melo Santos Velloso \\{\small MSc. student}}

\institute{Gleb Wataghin Institute of Physics - University of Campinas\\ Accelerator Physics Group (FAC) -  Brazilian Syncrhotron Laboratory (LNLS)}

\date{Optics Tuning and Corrections for Future Colliders Workshop \\ CERN, June 2023}

\AtBeginSection[]
{
    \begin{frame}
         \tableofcontents[currentsection]
   \end{frame}
}

\begin{document}
\maketitle
\begin{frame}{Contents}
    \tableofcontents
\end{frame}

\section{Introduction}
\begin{frame}{SIRIUS storage ring}
    \begin{minipage}{0.35\textwidth}
        \begin{figure}
            \centering
            \includegraphics[angle=90, width=0.7\textwidth]{f1.png}
        \end{figure}
    \end{minipage}
    \hfill
    \begin{minipage}{0.62\textwidth}
        \scriptsize
        Designed, built and operated by the Brazilian Synchrotron Laboratory (LNLS), at the Brazilian Center for Research in Energy and Materials (CNPEM) campus, at Campinas, Brazil.\\

        \begin{tabular}{lccc}
                \toprule\toprule
                Parameter & & Currently & Phase I \\
                \toprule
                Energy  & $E_0$  & $\SI{3}{\giga\electronvolt}$ & \\
                Current & $I_0$ &  $\SI{100}{\milli\ampere}$ & $\SI{350}{\milli\ampere}$ \\
                Operation mode & & Top-up &      \\
                RF Cavities & & 1 NC & 2 SC + HC \\
                RF Voltage & $\hat{V}_{\mathrm{rf}}$ &  $\SI{1.5}{\mega\volt}$ & $\SI{3.0}{\mega\volt}$\\
                RF Frequency &   $f_{\mathrm{rf}}$ &  $\SI{499.667}{\mega\hertz}$ &  \\
                Harmonic Number &   $h$ &  864 \\
                Momentum compaction factor & $\alpha$ &   $\SI{1.6e-4}{}$ & \\
                Energy Spread & $\sigma_\delta$ &  $\SI{8.5e-4}{}$ & \\
                Bunch length & $\sigma_z$ &  $\SI{2.5}{\milli\meter}$ & $\SI{12}{\milli\meter}$ \\
                Energy loss p/ turn & $U_0$ &  $\SI{470}{\kilo\electronvolt}$ & $\SI{870}{\kilo\electronvolt}$ \\
                Lifetime & $\tau$ & $\SI{15}{\hour}$ & $>\SI{10}{\hour}$ \\
                % Synchrotron Tune & $\nu_z$ & $\SI{0.001638}{}$ \\
                % Frequência Síncrotron & $f_z$ &  $\SI{2.5}{\kilo\hertz}$ & \\
                % Amortecimento Longitudinal & $\tau_\delta$ &  $\SI{13}{\milli\second}$ & \\             % \\
                % \bottomrule
                % HC Type &  & Passive SC \\
                % HC RF harmonic & $q$ & 3 \\
                % HC Shunt Impedance & $R_s$ & $\SI{8.25}{\mega\ohm}$ \\
                % HC Quality Factor & $Q$ & $\SI{20800}{}$ \\
                % HC R/Q & $R/Q$ & $\SI{396}{\ohm}$ \\
                \bottomrule\bottomrule
        \end{tabular}
    \end{minipage}
\end{frame}

\begin{frame}{SIRIUS Lattice and Optics}
    20-cell 5BA lattice with 5-fold symmetric high (A) and low (B, P) betatron functions sections
    \begin{figure}
        \centering
        \includegraphics[width=0.8\textwidth]{SI_superperiod.png}
    \end{figure}
    \begin{figure}
        \includegraphics[width=0.8\textwidth]{linear_optics.pdf}
    \end{figure}
\end{frame}

\section{Online tuning of storage ring non-linear dynamics}
\begin{frame}{Off-axis injection scheme}
    \begin{figure}
        \centering
        \includegraphics[width=0.7\textwidth]{injection.pdf}
    \end{figure}
    \begin{figure}
        \centering
        \includegraphics[width=0.6\textwidth]{nlk_phase_space.png}
    \end{figure}
\end{frame}
\begin{frame}{Tuning at $\nu_x = 49.08, \nu_y = 14.14$ (Working Point 1)}
    \begin{minipage}{0.55\textwidth}
        \begin{figure}
            \centering
            \includegraphics[width=\textwidth]{oldtunes_history.pdf}
            \includegraphics[width = 0.75\textwidth]{WEPL087_f1.pdf}
            \vfill
            \scriptsize
            \begin{table}[]
                \begin{tabular}{cc}
                \hline
                configuration & injection efficiency $[\%]$ \\ \hline
                ref-config    & $88\pm8$                    \\
                run 1         & $91\pm1$                    \\
                run 2         & $98\pm1$                     \\
                run 3         & $87\pm3$                     \\ \hline
                \end{tabular}
                \end{table}
        \end{figure}
    \end{minipage}
    \hfill
    \begin{minipage}{0.44\textwidth}
        \begin{figure}
            \centering
            \includegraphics[height=0.9\textheight]{WEPL087_f2.pdf}
        \end{figure}
    \end{minipage}
\end{frame}
\begin{frame}{Tuning at $\nu_x = 49.20, \nu_y = 14.25$ (Working Point 2)}
    \begin{minipage}{0.55\textwidth}
        \begin{figure}
            \centering
            \includegraphics[width=\textwidth]{newtunes_history.pdf}
            \includegraphics[width = 0.75\textwidth]{WEPL087_f3.pdf}
            \begin{table}[]
                \scriptsize
                \begin{tabular}{cc}
                \hline
                configuration & injection efficiency $[\%]$  \\ \hline
                non-optimized    & $51\pm1$                   \\
                run 1            & $79\pm3$                    \\
                run 2            & $65\pm1$                     \\ \hline
                \end{tabular}
                \end{table}
        \end{figure}
    \end{minipage}
    \hfill
    \begin{minipage}{0.44\textwidth}
        \begin{figure}
            \centering
            \includegraphics[height=0.9\textheight]{WEPL087_f4.pdf}
        \end{figure}
    \end{minipage}
\end{frame}
\begin{frame}{Tuning at $\nu_x = 49.16, \nu_y = 14.22$ (Working Point 3)}
    \begin{minipage}{0.55\textwidth}
        \begin{figure}
            \centering
            \includegraphics[width=\textwidth]{wp3_history.pdf}
            \includegraphics[width = 0.75\textwidth]{wp3_kick_resilience.pdf}
            \begin{table}[]
                \scriptsize
                \begin{tabular}{cc}
                \hline
                configuration & injection efficiency $[\%]$  \\ \hline
                non-optimized       & $-\pm1$                   \\
                optimied            & $93\pm3$                    \\ \hline
                \end{tabular}
                \end{table}
        \end{figure}
    \end{minipage}
    \hfill
    \begin{minipage}{0.44\textwidth}
        \begin{figure}
            \centering
            \includegraphics[height=0.9\textheight]{wp3_phase_space.pdf}
        \end{figure}
    \end{minipage}
\end{frame}

\begin{frame}{Summary}
    \begin{itemize}
    \item Tuning was effective at optimizing injection efficiency
    \item Some mysteries
    \begin{itemize}
        \item Larger kick resiliency $\centernot\implies$ larger phase portrait areas $\centernot\implies$ injection efficiency
    \end{itemize}
    \item A good sextupole setting was found in WP 3, which contributed for SIRIUS recent milestone of reaching $<1\% \sigma_x$ and $<4\% \sigma _y$ orbit stability in the horizontal and vertical, respectively
    \end{itemize}
\end{frame}

\section{Fast ORM Measruement}
\end{document}
